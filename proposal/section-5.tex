%\section{Theoretical Framework} \label{methods}
\section{Methodology} \label{methods}

A research problem should, where possible, be set within the framework of a
theory. A ``theory'' is a collection of interrelated law-like statements or
hypotheses aimed at explaining a phenomenon. Theories suggest hypotheses to be
tested. A hypothesis is a conjectural, conditional (if-then) statement linking
two or more variables. Hypothesis grows out of theoretical or conceptual
frameworks. The theoretical or conceptual framework and the resultant
hypotheses will identify and name the important variables to be studied. The
student must identify the variables and define the variables or terms
conceptually and operationally.
